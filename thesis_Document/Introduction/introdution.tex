\chapter{Introduction}
\label{sec:introduction} 
In the rapidly evolving landscape of \ac{VR}, spearheaded by industry behemoths such as Meta~\cite{bezmalinovic2022meta}, Microsoft~\cite{microsoft2022annual}, and Apple~\cite{savitz2023apple}, who are investing billions in immersive technologies, we are witnessing a transformative shift. This shift transcends the physical constraints of our world, opening up a realm of limitless possibilities. The advancements in \ac{GPU} ~\cite{alba2023gpus} and computing capabilities, along with powerful gaming engines, have brought us to a crucial crossroads on the path to a fully immersive virtual existence. The level of progress we have reached in this field was beyond imagination just a few decades ago.\\ \\
In our understanding of reality, sensation plays a foundational role as it involves receiving and interpreting information from our surroundings through our sensory systems. Humans and their environments are inherently designed with biological interfaces that facilitate direct interaction, allowing sensations to be translated seamlessly without the need for mediators. This intrinsic connection between individuals and their environments forms the basis of our perceptual experience, where each sensory input is integrated and acted upon holistically~\cite{goldstein2016sensation}.\\ \\ In the realm of \ac{VR}, this direct translation of sensation presents both a challenge and an opportunity. \ac{VR} seeks to replicate this seamless interface in a digital context, aiming to create environments where users can interact with virtual objects as intuitively as they would with physical ones. The fidelity of these interactions in \ac{VR} hinges on the ability to accurately simulate and elicit natural human responses to virtual stimuli, making the study of human sensory and perception systems critical to advancing \ac{VR} technologies~\cite{steuer1992defining, slater2020immersion, lanier2006homuncular, bowman2007virtual, heeter1992being}. 

\section{Motivation}
\label{sec:Motivation}
Given this groundbreaking shift, this new paradigm presents a meta-problem: how do we
translate the complex language of human sensation and perception~\cite{goldstein2016sensation} into the virtual domain? The interface between humans and computers, especially in immersive environments
such as \ac{VR}, poses significant challenges due to the need to accurately
replicate sensory inputs so that they mimic the natural interactions one would expect in the
real world ~\cite{sheridan1992musings,slater2018implicit}. One area grappling with this challenge is software development, where
text entry remains a fundamental skill for coding and debugging. The keyboard, a historical
mainstay in programming, continues to be an indispensable tool. Despite explorations into
voice recognition and gesture-based interfaces, efficient typing is still closely tied to coding
proficiency~\cite{dahl2018text}. Yet, the challenge becomes even more pronounced in \ac{VR}, where the tactile
and precise nature of traditional keyboards is disrupted, and each \ac{HMD} supplier offers isolated solutions without standardization ~\cite{kruger2018text}.\\ \\
In \ac{VR}, not only is the physical interaction with keyboards transformed but the perceptual alignment of visual, tactile, and kinesthetic cues is also critical for user proficiency and
comfort. The lack of a standard approach complicates the development of universal solutions
that could benefit all users across different systems. Moreover, the quest for efficient
text entry in \ac{VR} is further complicated by the diversity of user backgrounds and ergonomic
needs, which require adaptable and flexible input methods~\cite{bowman2007virtual}. This makes it essential for \ac{VR} platforms to evolve beyond mere replication of the physical tools and to innovate towards more intuitive and integrated input methods that cater to the needs of diverse user
populations.

\section{Research Goals}
\label{sec: Research Goals} 
This literature review aims to tackle the specific issue of enhancing text entry in \ac{VR}, a fundamental challenge that bridges human-computer interaction and virtual environment design. Given the rapid typing speeds and extended daily keyboard use by programmers, accuracy issues in existing virtual reality tracking systems are a significant concern. The review methodically delves into existing research, identifying gaps in current methodologies and synthesizing insights that could shape the future trajectory of work in enhancing user experience and productivity within the virtual reality programming sphere. The goal extends beyond merely replicating the physical keyboard in a virtual space. Instead, it seeks to leverage the unique properties of \ac{VR}—such as spatial interaction, multimodal integration, and context-aware interfaces—to create a more efficient and intuitive typing experience. \\ \\
Innovations in \ac{VR} text entry are poised to transform how users interact with software in immersive environments, facilitating smoother and more natural interactions that could significantly boost coding and debugging efficiency. By integrating insights from fields such as ergonomics, cognitive psychology, and artificial intelligence, this research aims to develop a set of design principles and technical guidelines that will inform the development of next-generation \ac{VR} keyboards. These enhanced interfaces are expected not only to improve the speed and accuracy of text entry but also to reduce the cognitive load on users, thereby making \ac{VR} more accessible and appealing for a broader range of professional and recreational applications.\\ \\
By addressing the challenges of text entry in \ac{VR}, this work will contribute to the overarching goal of making virtual environments as functional and user-friendly as their physical counterparts, thereby paving the way for more widespread adoption and innovative uses of \ac{VR} technology.
