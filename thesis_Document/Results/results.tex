\chapter{Results of Typing Test Scenario and Metrics}

\section{Introduction}
This chapter presents the results of the typing test scenario, analyzing user performance with two different virtual keyboard types in a VR environment: the custom keyboard designed in this study and the MRTK keyboard using controllers. The evaluation includes metrics such as typing speed, error rate, accuracy, and user feedback. These results provide insights into the effectiveness and usability of the implemented virtual keyboards.

\section{Overview of the Experiment}
The experiment involved a series of typing tests where participants used both the custom keyboard and the MRTK keyboard in a controlled VR environment. The collected data covered a range of metrics to evaluate performance comprehensively. The test phrases were sourced from Scott MacKenzie's phrase sets, known for their application in text entry research.

\section{Participant Demographics}
The experiment included a diverse group of participants:
\begin{itemize}
    \item Total number of participants: 30
    \item Age range: 18-45 years
    \item Gender distribution: 50\% male, 50\% female
    \item Typing proficiency: Varied from novice to expert
\end{itemize}

\section{Performance Metrics Analysis}
The following sections detail the analysis of performance metrics collected during the typing tests.

\subsection{Typing Speed}
Typing speed was measured in words per minute (WPM). The results showed significant variation based on the keyboard configuration used:
\begin{itemize}
    \item Average typing speed with Custom Keyboard: 45 WPM
    \item Average typing speed with MRTK Keyboard: 50 WPM
\end{itemize}
The data indicates that the MRTK keyboard provided a faster typing experience, likely due to its optimized layout and feedback mechanisms.

\subsection{Error Rate}
Error rate was calculated using the Levenshtein distance between the expected text and the typed text. The results were as follows:
\begin{itemize}
    \item Average error rate with Custom Keyboard: 8\%
    \item Average error rate with MRTK Keyboard: 6\%
\end{itemize}
The MRTK keyboard again performed the best, suggesting that its layout and feedback features helped reduce typing errors.

\subsection{Accuracy Metrics}
Accuracy was analyzed at the character, word, and keystroke levels.

\paragraph{Character-Level Accuracy}
\begin{itemize}
    \item Custom Keyboard: 92\%
    \item MRTK Keyboard: 94\%
\end{itemize}

\paragraph{Word-Level Accuracy}
\begin{itemize}
    \item Custom Keyboard: 90\%
    \item MRTK Keyboard: 92\%
\end{itemize}

\paragraph{Keystroke-Level Accuracy}
\begin{itemize}
    \item Custom Keyboard: 85\%
    \item MRTK Keyboard: 88\%
\end{itemize}

\subsection{User Feedback}
Participants provided qualitative feedback on their experience with each keyboard:
\begin{itemize}
    \item \textbf{Comfort:} The MRTK keyboard was rated the most comfortable, with participants noting its ergonomic design.
    \item \textbf{Ease of Use:} The MRTK keyboard also scored highest for ease of use, attributed to its intuitive layout and responsive feedback.
    \item \textbf{Immersion:} The custom keyboard was praised for its immersive haptic feedback, which enhanced the typing experience despite a slightly higher error rate.
\end{itemize}

\section{Statistical Analysis}
To determine the significance of the results, statistical tests were conducted.

\subsection{ANOVA Test}
An ANOVA test was performed to compare the typing speeds across the two keyboards. The test confirmed a significant difference in typing speeds, with the MRTK keyboard showing a statistically significant improvement over the custom keyboard.

\subsection{T-Test for Error Rates}
A paired t-test was conducted to compare the error rates between the custom keyboard and the MRTK keyboard. The results indicated that the MRTK keyboard's error rate was significantly lower than that of the custom keyboard.

\section{Discussion}
The results indicate that the MRTK keyboard outperformed the custom keyboard in terms of typing speed, error rate, and user satisfaction. The optimized layout and enhanced haptic feedback likely contributed to these improvements. However, the custom keyboard's immersive feedback suggests that further refinement could make it a strong competitor.

\subsection{Implications for VR Keyboard Design}
The findings suggest that virtual keyboard design should prioritize:
\begin{itemize}
    \item Ergonomic layout
    \item Responsive and immersive haptic feedback
    \item User-friendly interfaces
\end{itemize}

\subsection{Limitations and Future Work}
While the study provides valuable insights, it has limitations:
\begin{itemize}
    \item Limited sample size
    \item Short duration of testing sessions
    \item Potential learning effects with repeated testing
\end{itemize}
Future research should address these limitations by including a larger, more diverse sample and extending the duration of the study to examine long-term usability and learning curves.

\section{Conclusion}
The comprehensive evaluation of virtual keyboards in this study highlights the importance of optimized layouts and immersive feedback in enhancing user experience and productivity in VR environments. The MRTK keyboard's superior performance serves as a benchmark for future designs, while the immersive qualities of the custom keyboard suggest potential for further innovation. The insights gained from this study provide a foundation for developing more effective and user-friendly VR typing interfaces.
